\documentclass[11pt, a4paper]{article}


\usepackage[T1]{fontenc} 
\usepackage[utf8]{inputenc} 
\usepackage{polski} 
\usepackage{geometry} 
\usepackage{graphicx} 
\usepackage{hyperref} 
\usepackage{listings} 
\usepackage{amsmath} 
\usepackage{amssymb} 


\geometry{
    a4paper,
    left=2.5cm,
    right=2.5cm,
    top=2.5cm,
    bottom=2.5cm
}


\lstset{
    language=SQL,
    basicstyle=\ttfamily\small,
    keywordstyle=\bfseries\color{blue},
    commentstyle=\color{green!60!black},
    stringstyle=\color{red},
    showstringspaces=false,
    breaklines=true,
    frame=single,
    tabsize=4,
    captionpos=b
}


\title{Projekt Bazy Danych: System Zarządzania Koleją}
\author{Jakub Durczyński, Błażej Szeląg, Pavlo Tsikalyshyn} 
\date{Kwiecień 2025} 


\begin{document}

\maketitle

\section{Tematyka i cele projektu}

Projekt dotyczy systemu bazodanowego do zarządzania transportem kolejowym. Celem jest stworzenie bazy danych przechowującej informacje o stacjach, torach, trasach, pociągach (wraz z ich składem i typami), klientach, biletach, rezerwacjach miejsc oraz rozkładach jazdy. System powinien zapewniać spójność danych i umożliwiać efektywne zarządzanie operacjami kolejowymi.

\section{Opis schematu bazy danych}

Schemat bazy danych modeluje kluczowe elementy systemu kolejowego. Główne grupy tabel obejmują:
\begin{itemize}
	\item \textbf{Infrastruktura:} Tabele \texttt{stacje}, \texttt{tory}, \texttt{polaczenia\_miedzy\_stacjami} opisują fizyczną infrastrukturę kolejową.
	\item \textbf{Trasy i Rozkład:} Tabele \texttt{trasa}, \texttt{stacje\_na\_trasie}, \texttt{rozklad\_jazdy}, \texttt{odcinek} definiują logiczne trasy, ich realizację w czasie oraz powiązanie z pociągami.
	\item \textbf{Pojazdy kolejowe:} Tabele \texttt{lokomotywy}, \texttt{typy\_pociagow}, \texttt{pociagi}, \texttt{typy\_wagonu}, \texttt{wagony}, \texttt{sklad\_pociagu}, \texttt{miejsce} zarządzają informacjami o pojazdach kolejowych, ich typach, składach i dostępnych miejscach.
	\item \textbf{Klienci i Bilety:} Tabele \texttt{klient}, \texttt{znizka}, \texttt{bilet}, \texttt{wykupione\_miejsca} obsługują dane klientów, dostępne zniżki, sprzedane bilety oraz rezerwacje konkretnych miejsc na dane odcinki.
\end{itemize}

\section{Napotkane problemy i rozwiązania}

Podczas projektowania napotkano następujące wyzwania:
\begin{itemize}
	\item Efektywne modelowanie złożonych relacji między trasami, rozkładami jazdy i fizycznymi odcinkami pokonywanymi przez pociągi.
	\item Zapewnienie unikalności rezerwacji miejsc na poszczególnych odcinkach podróży.
	\item Wyszukiwanie połączeń między stacjami: Zaprojektowanie mechanizmu wyszukiwania podróży między dwoma stacjami, które nie muszą mieć bezpośredniego połączenia. Wymaga to implementacji algorytmu przeszukiwania grafu, np. algorytmu Dijkstry, do znalezienia optymalnej trasy (np. najkrótszej lub najszybszej) z uwzględnieniem możliwych przesiadek.
\end{itemize}
Rozwiązanie tych problemów wymagało odpowiedniego zaprojektowania relacji oraz kluczy głównych i obcych.
Implementacja systemu wyszukiwania połączeń i optymalnej trasy, zaplanowane jest na dalszą część projektu.

\section{Planowane dalsze prace}

W kolejnych etapach planowane jest dodanie:
\begin{itemize}
	\item \textbf{Indeksów} w celu optymalizacji zapytań.
	\item \textbf{Perspektyw} dla uproszczenia dostępu do złożonych danych (np. rozkład dla stacji, dostępne miejsca).
	\item \textbf{Funkcji} do np. kalkulacji ceny biletu, wyszukiwania połączeń z wykorzystaniem algorytmu Dijkstry.
	\item \textbf{Wyzwalaczy} do automatyzacji zadań i utrzymania spójności (np. aktualizacja statusu pociągu, kontrola liczby miejsc).
\end{itemize}

\end{document}

